% Please do not change the document class
\documentclass{scrartcl}

% Please do not change these packages
\usepackage[hidelinks]{hyperref}
\usepackage[none]{hyphenat}
\usepackage{setspace}
\doublespace

% You may add additional packages here
\usepackage{amsmath}
\usepackage{graphicx}
\usepackage{url}
\usepackage[autostyle]{csquotes}

% Please include a clear, concise, and descriptive title
\title
{Research Journal}

% Please do not change the subtitle
\subtitle{COMP210}

% Please put your student number in the author field
\author{Student number: 1607539}

\begin{document}

\maketitle

\blockquote {Saying that VR is isolating because it's immersive is a really narrow view of the world. The reality is we all have limits to our reality. And opening up more of those experiences to all of us, that's not isolating. That's freeing. - Mark Zuckerberg}\\
\indent\blockquote {Virtual reality is a denial of reality. We need to be open to the powers of imagination, which brings something useful to reality. Virtual reality can imprison people. - Hayao Miyazaki}

\section*{\underline{Virtual and Augmented reality}}

Mixed reality (MR) is where real and virtual objects are displayed together \cite{milgram1994taxonomy} and the designs and applications mixing these two qualities fall somewhere along a virtuality continuum \cite{venta2014investigating}. Virtual reality (VR) and augmented reality (AR) technology has been primarily invented for games, but can have huge applications in other sectors such as training, education \cite{mentzelopoulos2015hardware}, sports \cite{azuma2001recent}, and medicine - helping to reduce post-traumatic stress disorder \cite{rothbaum2001virtual}. It is best used anywhere that traditional mouse and keyboard interaction is not suitable such as mobile and wearable technology \cite{keir2006gesture}. 

Users are receptive to using VR as a learning tool \cite{blackledge2012evaluation}, but the technology is still progressing to a state where it can be commonly used \cite{huang2013mobile}. It seems to me that the smaller the head mounted displays (HMD) become; the desire for even smaller devices grows. As HMD's now possess dual 1080 globally refreshing screens at 90Hz, consumers look more to adopting their phones as cheaper alternatives and questioning why the experience isn't perfect?

Whilst there is concern that virtual reality would take away the authentic, concrete real world experience and alienate the user from it \cite{venta2014investigating}, practically the latest virtual environments (VE) require less preparation time and resources, and provide an advantage of testing hypothetical situations without disruption in a community \cite{hvannberg2012exploitation}. Fire and safety training can be augmented with heat and noise output to try and simulate the real world stresses that may occur, and the risks of a trainee learning to drive a large vehicle can be completely negated.

It may be that at some point humans can no longer tell the difference between what is and what is not physical in the perceptual sense, meaning the experiences for real, mediated, and virtual could all be identical \cite{rosa2016re}. Although, some users already consider a computer simulated view to be clearer and more usable than the plain camera view \cite{venta2014investigating}, seeing the virtual representation as a filtered view of the real world. Information can be readily more apparent without the distraction of say a very busy street when looking for a particular store.

As a learning tool, the interface must have excellent mapping to better complete the mental model for how to perform the real-life actions \cite{skalski2011mapping}, and the entire experience relies heavily on presence. In VR this is when a person behaves and responds as if they are in the place represented by the virtual environment \cite{bianchi2013understanding}. Outside triggers must be kept to minimum with well fitting HMDs that offer a wide field of view, and the technology used must be of an excellent standard to avoid simulation sickness caused by latency and lag from the system \cite{biocca1997cyborg} \cite{steinicke2014self}, or an overly limited gesture space \cite{walter2000incremental}. This is where I fear that VR may lose a lot of respect in a world of consumers, many of whom's only experience is sub par from using a phone with a cheap adapter.

Once MR is common place, there is a miriad of options available to develop. Studies show that neither the perceived body size nor shape is as rigid as we may believe \cite{kilteni2012extending}. Even though watching a film does not create the strong illusion of owning and controlling a body that is not your own \cite{madary2016real}, simply perceiving a virtual body from a first-person perspective can create a sense of identification with an avatar \cite{won2015homuncular}, often even the persona of the avatar \cite{yoon2014know}. VR can move past simulating events, such as driving a formula one car, and truly envelope the user in a new experience.

Mobile augmented reality \cite{de2012mobile} is commonly used with maps and direction finding programs, finding shop opening hours and bus timetables \cite{venta2012user}, but doesn't have to solely visual \cite{kalawsky2000taxonomy}. Mixing all the modalities \cite{lindeman2007classification} can offer additionally olfactory or auditory enhancements. A vibration belt has even been used as a haptic guide when visibility is impaired \cite{van2005waypoint}. It is important to note that MR applications are not just for entertainment. That they can be serious and save lives.

Without a traditional mouse and keyboard and to preserve engagement, intuitive controls are not always agreed upon \cite{nielsen2003procedure} unless just basic movements like left right up down \cite{oskoei2009application}. Many users find it difficult to explain what contributes to a successful 3D interface \cite{john1998traditional}. Functionality over aesthetic is a choice which threatens the very immersive nature of MR. Intuitive movements, currently without an industry standard, are personal without any right or wrong answer, therefore researching head tracking \cite{azuma1993tracking} and the practical matter of performing actual evaluations can be quite difficult \cite{bowman2002survey}. Often the emotional experiences of fascination and amazement can be just due to the charm of a novel technology \cite{olsson2012narratives}, an industry recognised standard walkthrough method for evaluating virtual reality usability assessment, goal-orientated tasks, exploration and navigation, and interaction in response to system initiative is needed \cite{sutcliffe2000evaluating}.


\bibliographystyle{ieeetran}
\bibliography{references}

\end{document}

This technology has been primarily invented for games, but it will have huge applications in other sectors such as training, education and illustrations \cite{mentzelopoulos2015hardware}

Adding ``value'' to the users' ears. This is the main reason for the appeal of stereophonic sound. Just as two visual perspectives make a 3D view, two audio perspectives can make a 3D soundscape. However, with free-standing stereo speakers the left and right sounds are mixed: both ears hear sound from both speakers. By using headphones and presenting the correct acoustical perspectives to each ear, many of the spatial aspects of sounds can be preserved. HMDs often have headphones built into them \cite{mentzelopoulos2015hardware}

Current training methods (e.g. large-scale, multiagency, real-life exercises) for crisis management bring a considerable amount of personnel and organizational complexity that implies significant financial and temporal demands. The latest virtual environments (VE) can provide high fidelity, require less preparation time and resources, and provide an advantage of testing hypothetical situations without a disruption in a community \cite{hvannberg2012exploitation}

The evaluation difficulty was considered to lie in the absence of analysis tools for 3D user interface interaction and measurement standards \cite{hvannberg2012exploitation}

not a single heuristics list may be useful to help evaluators uncover problems, but a patchwork of heuristics \cite{hvannberg2012exploitation}

researchers do not seem to agree on how generic or specific such heuristics should be \cite{hvannberg2012exploitation}

mixed reality as the general term for merging of real and virtual worlds, as it is defined by Milgram & Kishino [17] where the designs and applications mixing these two qualities fall somewhere along the virtuality continuum \cite{venta2014investigating}

concern that virtual reality would take away the authentic, concrete real world experience and alienate the user from it \cite{venta2014investigating}

The 3D city model was well recognized to present the same location, and it was generally considered clearer and more usable than the plain camera view \cite{venta2014investigating}

suggest that people can be led to alter their body schema by sensory input \cite{won2015homuncular}

simply perceiving a virtual body from a first-person perspective can create a sense of identification with an avatar \cite{won2015homuncular}

Virtual reality lends itself to dramatic explorations of the types of bodies that humans can learn to inhabit and control \cite{won2015homuncular}

it may be that at one point humans can no longer tell the difference between what is and what is not physical in the perceptual sense, meaning the experiences for real, mediated, and virtual could all be identical \cite{rosa2016re}

watching a film or playing a non-immersive video game cannot create the strong illusion of owning and controlling a body that is not your own \cite{madary2016real}

ambiguity raises the likelihood that users may give consent for data collection in a particular virtual context but then become unaware of the continued data collection as the user changes context \cite{madary2016real}
